%\VignetteIndexEntry{Introduction to Defaults}
\documentclass{article}

\title{\bf Introduction to Defaults }
\author{Jeffrey A. Ryan \\
       jeff.a.ryan@gmail.com}
\date{July 25, 2007}

\usepackage{/export/home/jryan/Desktop/R-2.5.0/share/texmf/Sweave}
\begin{document}

\maketitle

\section{Introduction}
A common problem when writing functions
for use by others is deciding upon sensible
default values that will appeal to most of
your target user's daily  needs.

\vspace{5mm}
\noindent
As an end user, often these defaults may not fit
the problem at hand, and will require some fine-tuning
to make the function perform as desired.  Additionally,
while all 
values may be changed within an actual
function call, it may not always be desirable to
have to remember the new defaults, or to re-enter
them with each function call.

\vspace{5mm}
\noindent
The {\sf Defaults}~\cite{p} package allows the end user
to pre-specify a default for any formal argument
in a given function, and to {\em force} the function
to use these defaults. For the function authors it
is no longer necessary to hand-code checks to
{\sf R}'s internal {\sf options}, as the addition
of one function call at the beginning of any function
needing access to user specified defaults will 
manage the process for them. 

\vspace{5mm}
\noindent
This document will cover the {\sf Defaults} package
from the perspective of the user and the developer. We begin
with how the {\sf R} end user can benefit from using
{\sf Defaults}.


\section{The End User}
Every person using R, whether for analysis or as a
developer, is an end user. Countless functions
are used within a typical session, often with
multiple optional argument settings for each.
One of the most common is {\tt ls}. If one would like
to have {\tt ls} display even hidden objects in a given environment
it is necessary to add the argument {\tt all.names=TRUE} to the call.
With {\sf Defaults}, one can specify outside the function call this
new default value, so that subsequent calls will now display all
names.

\begin{Schunk}
\begin{Sinput}
> hello <- "visible"
> .goodbye <- "hidden"
> ls()
\end{Sinput}
\begin{Soutput}
[1] "hello"
\end{Soutput}
\begin{Sinput}
> library(Defaults)
> setDefaults("ls", all.names = TRUE)
> useDefaults(ls)
> ls()
\end{Sinput}
\begin{Soutput}
[1] ".goodbye" "hello"    "ls"      
\end{Soutput}
\begin{Sinput}
> ls(all.names = FALSE)
\end{Sinput}
\begin{Soutput}
[1] "hello" "ls"   
\end{Soutput}
\end{Schunk}

\noindent
After loading the {\sf Defaults} library, a call to
{\tt setDefaults('ls',all.names=TRUE)} is made. This
creates an entry in the standard {\sf options} list, with
the name {\tt ls.Default}, attaching the value all.names=TRUE
to this entry.

\vspace{5mm}
\noindent
At this point the original function {\tt ls} is unable to
process this new user specified default. By calling
{\tt useDefaults(ls)} a new copy of {\tt ls} is added
to the user's workspace, with the notable difference that this
copy {\em can} process the new defaults. The original function is
effectively hidden from the user, allowing the {\sf Defaults}
functionality to be used. In the event that the function is
{\em already} in the user's workspace (e.g. a user defined function)
a copy of the original is made and hidden from normal view.

\vspace{5mm}
\noindent
Internally, the function first looks to see if
any arguments have been specified in the actual
function call, as these take precedence over {\em any}
default - formal or via {\tt setDefaults}. If no
value is given in the call, the global defaults, if any,
are checked.  If nothing is still set, the process falls
back to the original formal defaults, if any, and continues
executing.

\vspace{5mm}
\noindent
At present it is {\em NOT} possible to set an
argument's value as {\tt NULL} or to a function.
Values that cannot be set via {\tt setDefaults} may
of course still be specified in the function call.

\vspace{5mm}
\noindent
The set defaults can be viewed with {\tt getDefaults}, and unset with
a call to {\tt unsetDefaults}.  The former, when called with no
arguments, will return a character vector of all functions currently
having defaults set for use with {\sf Defaults}.

\begin{Schunk}
\begin{Sinput}
> getDefaults(ls)
\end{Sinput}
\begin{Soutput}
$name
NULL

$pos
NULL

$envir
NULL

$all.names
[1] TRUE

$pattern
NULL
\end{Soutput}
\begin{Sinput}
> getDefaults()
\end{Sinput}
\begin{Soutput}
[1] "ls"
\end{Soutput}
\begin{Sinput}
> unsetDefaults(ls, confirm = FALSE)
\end{Sinput}
\end{Schunk}

\vspace{2mm}
\noindent
Since using global default as an end user is so easy,
it only makes sense that making use of them as a
developer would be just as straighforward. It's even easier.

\section{The Developer}
Without the {\sf Defaults} package, if one is to use
a mechanism to access globally specified defaults, designed
specifically for a new function, it would require
a complete lookup facility, as well as a series of {\tt if-else}
blocks.  With {\sf Defaults} all that is required is one function
placed at the beginning of your function.

\begin{Schunk}
\begin{Sinput}
> fun <- function(x = 5, y = 5) {
+     importDefaults()
+     x^y
+ }
> fun()
\end{Sinput}
\begin{Soutput}
[1] 3125
\end{Soutput}
\begin{Sinput}
> fun(x = 10)
\end{Sinput}
\begin{Soutput}
[1] 1e+05
\end{Soutput}
\end{Schunk}

\vspace{2mm}
\noindent
{\tt importDefaults()} places all {\it non}-NULL default
arguments specified by an earlier call to 
{\tt setDefaults} into the current function's environment.
The only exception would be if the argument has been specified
in the function call itself, at which point the value or values
in question would {\em NOT} be loaded into the current scope.

\begin{Schunk}
\begin{Sinput}
> setDefaults(fun, x = 8, y = 2)
> fun()
\end{Sinput}
\begin{Soutput}
[1] 64
\end{Soutput}
\begin{Sinput}
> fun(9)
\end{Sinput}
\begin{Soutput}
[1] 81
\end{Soutput}
\begin{Sinput}
> fun(y = 0.5)
\end{Sinput}
\begin{Soutput}
[1] 2.828427
\end{Soutput}
\begin{Sinput}
> unsetDefaults(fun, confirm = FALSE)
> fun()
\end{Sinput}
\begin{Soutput}
[1] 3125
\end{Soutput}
\end{Schunk}

\section{Conclusion}
Using {\sf Defaults}, whether as an end user or package developer,
greatly simplifies the process of utilizing externally set global
defaults. With a small set of functions, users can create and use
default arguments in place of formal ones, as well as create
defaults where none normally exist, all without relying on the underlying
function's own methods for handling defaults. Future development
may include the ability to use {\tt NULL} as a legal default for
the rare occasion that it is desired, as well as a better method of
handling subsequent function calls within the visible parent function,
as is the case with {\tt S3}-style method dispatch.

\begin{thebibliography}{99}
\bibitem{p} Jeffrey A. Ryan:
\emph{Defaults: Create Global Function Defaults},
R package version 1.0-5, 2007
\end{thebibliography}
\end{document}
